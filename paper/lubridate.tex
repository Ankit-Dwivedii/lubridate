\documentclass[article]{jss}

%%%%%%%%%%%%%%%%%%%%%%%%%%%%%%
%% declarations for jss.cls %%%%%%%%%%%%%%%%%%%%%%%%%%%%%%%%%%%%%%%%%%
%%%%%%%%%%%%%%%%%%%%%%%%%%%%%%

%% almost as usual
\author{Garrett Grolemund\\Rice University \And 
        Hadley Wickham\\Rice University}
\title{Dates and Times Made Easy with \pkg{lubridate}}

%% for pretty printing and a nice hypersummary also set:
\Plainauthor{Garrett Grolemund, Hadley Wickham} %% comma-separated
\Plaintitle{Dates and Times Made Easy with lubridate} %% without formatting
%\Shorttitle{A Capitalized Title} %% a short title (if necessary)

%% an abstract and keywords
\Abstract{
  This paper presents the lubridate package for \proglang{R}, which facilitates working with dates and times.   
}
\Keywords{dates, times, time zones, daylight savings time, manipulation, \proglang{R}}
\Plainkeywords{dates, times, time zones, daylight savings time, manipulation, R} %% without formatting
%% at least one keyword must be supplied

%% publication information
%% NOTE: Typically, this can be left commented and will be filled out by the technical editor
%% \Volume{13}
%% \Issue{9}
%% \Month{September}
%% \Year{2004}
%% \Submitdate{2004-09-29}
%% \Acceptdate{2004-09-29}

%% The address of (at least) one author should be given
%% in the following format:
\Address{
  Garrett Grolemund\\
  Rice University\\
  Houston, TX 77251-1892, United States of America\\
  E-mail: \email{grolemund@rice.edu}
}
%% It is also possible to add a telephone and fax number
%% before the e-mail in the following format:
%% Telephone: +43/1/31336-5053
%% Fax: +43/1/31336-734

%% for those who use Sweave please include the following line (with % symbols):
%% need no \usepackage{Sweave.sty}

%% end of declarations %%%%%%%%%%%%%%%%%%%%%%%%%%%%%%%%%%%%%%%%%%%%%%%


\begin{document}
\section{Introduction}
Many data sets have a time component, which creates complicated problems for researchers. Unlike other measures, time is both exact and relative.  It is measured in units that have a precise length, such as seconds, as well as units whose length constantly changes, such as months. Variations in Daylight Savings Time and time zones further complicate time data.\\

Many \proglang{R} methods address these idiosyncrasies by creating new types of time objects.  But this approach makes research more complicated, not less. Moreover, the proliferation of time formats further complicates recognizing and accurately parsing time data.\\

\pkg{lubridate} simplifies working with time data in \proglang{R} by providing simple, intuitive solutions for manipulating date times, doing arithmetic with date-time objects, and reading in or parsing date-time objects. The package recognizes the dual nature of time and allows users to work with time as an exact measurement, a relative measurement, or both simultaneously. \pkg{lubridate} also offers simple methods for handling time zones and Daylight Savings Time(DST) and creates an improved user interface similar to object oriented programming languages.\\

This paper introduces the dual nature of time measurement, which will help you think about time related research problems, and demonstrates how \pkg{lubridate} can help you overcome those problems. The paper also demonstrates other convenient tools provided in the \pkg{lubridate} package and ends with a case study, using \pkg{lubridate} in a real life example.\\

\section{Relative time vs. specific time}
Time seems like an obvious example of an interval scale. Each moment of time, referred to as a date-time, is unique.  These moments occur in order, and each second represents the same amount of time no matter when it occurs.  Why then is working with time scale frustrating? To answer this question, it is helpful to recognize that time is an attempt to make two different measures coexist on the same scale. \\

In addition to measuring the passage of time, time also tracks both the rotation of the Earth and its revolution about the sun.  A day is meant to be the time length of one rotation of the Earth, and a year one revolution about the sun.  Further association is hinted at by the dual use of minutes and seconds as units of longitude.  These astronomical conditions are often more important than the exact time, but they rarely align with each other and do not occur with the precision of an interval scale.  The exact time has to be periodically re-calibrated to retain information about the Earth's astronomical position.  The leap year system is one example of this re-calibration. Many man made conventions, such as Daylight Savings Time, have also been invented in attempts to better align astronomical events and the passage of interval time.\\

In this way, two types of time coexist with each other: "exact" time that measures the interval progression of time, and "relative" time that measures the time relative to astronomical conditions and man-made conventions. Research involving time data may be interested in exact time, relative time, or both. The speed of a physical object is most precisely stated in exact time. The opening bids on most stock markets occur at the same relative time each day.  Wind speed, which is measured with exact time, may change depending on the time of the day, which is relative.\\ 

The coexistence of exact time and relative time creates many problems. Since, the Earth's angle of rotation appears different from one place to another, our labeling of time in different places varies accordingly. In other words, the same date-time will be given different labels at different locations on the globe. Alternatively, the same time label will correspond to different moments of date-time on different parts of the globe.  This a problem of identity.\\

Next, the re-calibrations and conventions made to pair relative time with exact time disrupt the consistency of many units of time.  A year beginning on January 1st, 2000 is 366 days long or 31622400 seconds, but a year beginning on January 1st, 2001 is only 365 days long or 31536000 seconds.  The length of a month can range from 28 days to 31 days depending on when the month begins. This presents a problem of duration, but it also poses a problem when we wish to convert between time units. How many hours does one day equal? What if that day is March 8th, 2009, the day when Daylight Savings Time goes into effect for parts of the world?
And this is another problem, how do we handle conventions like Daylight Savings Time that would affect relative time durations, such as the time from the market's close today until the opening tomorrow, but not exact durations such as the life of a lightbulb?\\

\section{Instants, Intervals, Durations, and Periods}

Instants are specific moments in exact time. Each instant lasts a millisecond, and instants occur in order. The uniform occurrence of instants creates the interval scale of time which we refer to as exact time. A single instant can be, and is, labeled in many different ways.  The most popular of these is clock time.\\

Clock time refers to the time that would be recorded on a clock at a given instant in a given place. Usually clock time is only an approximation of when a specific instance occurs. For example, most digital clocks do not display dates and round time down to the nearest minute.\\

Clock times vary from place to place, which makes them very consistent for measuring relative time.  7:00 am indicates one of the first hours of daylight no matter where you are on the planet. This also makes them poor indicators of instants. A single clock time could refer to many different instants depending on where it was measured.\\

The time zone system attempts to fix this.  By pairing a clock time with the time zone it occurred in, we can calculate the exact instant that clock time represents. When discussing exact time, it is standard to give the clock times that appear in the GMT or UTC time zones.  This saves calculations, but can be annoying if your computer insists on translating times to your current time zone.  It may also be inconvenient to discuss clock times that occur in a place unrelated to the data.\\

Even with the time zone system, clock time cannot accurately measure both exact time and relative time. The astronomical cycles measured by relative time do not perfectly align with each other, which is why we have a leap day almost every four years. Astronomical cycles also do not unfold as consistently as exact time. Aperiodic leap seconds are necessary to reconcile the rotation of the Earth, which is decelerating, with the progression of exact time, which is not.\\

In addition, clock time is sometimes adjusted to create a more convenient indication of relative time, such as when Daylight Savings Time occurs.  These adjustments are usually location specific. Months, which originally modeled lunar cycles, have also been adopted but with inconsistent lengths.\\

Because of all of this, the number of instants that have passed does not reliably tell us what changes have occurred in relative time. And likewise, a change in clock time does not reliably tell us how many instants have elapsed. In both cases we are missing a piece of information: when the change occurs. Since we know when inconsistencies between exact and relative time happen, we can use a starting time to calculate the instant and clock time that will result from elapsed time. This is similar to using a time zone to calculate what instant is being referred to by a given clock time.\\

\proglang{Joda-Time} defines such time span, instant pairs as intervals. Intervals differ from other time spans in that they are anchored by two known instants. Both intervals and non-interval time spans can be measured in exact time or relative time. Is the researcher interested in the exact number of 24 hour periods that occurred between 5:00 p.m. on Monday and 5:00 p.m. on Friday, or is she interested in the number of business days that transpired? These answers would be different if Daylight Savings Time changed in the interim.\\

We can evaluate an interval in either exact time or real time because it has been anchored to its starting time point. However, we must choose whether to define a floating time span in exact time or relative time.


\section{Inputting date-times}
\section{Working with date-times}
\subsection{Time zones}
\subsection{Daylight Savings Time}
\section{Durations}
\subsection{Object oriented interface}
\section{Case Study}
\section{Other convenience functions}
\section{Comparison table}
\section*{Acknowledgements}
\section*{References}


\section[About Java]{About \proglang{Java}}
%% Note: If there is markup in \(sub)section, then it has to be escape as above.

\end{document}
